%% Package
\usepackage{bxdpx-beamer}
\usepackage{pxjahyper}
\usepackage{minijs}
\usepackage[T1]{fontenc}
\usepackage{newpxtext, newpxmath}
\usepackage{tikz}
\usepackage{color}
\usepackage[deluxe]{otf}
\usepackage[ipaex]{pxchfon}
\usepackage{graphicx,xcolor}
\usepackage{listings,jvlisting}
\usepackage{multicol}
\usepackage[absolute,overlay]{textpos}
\definecolor{deepblue}{rgb}{0,0,0.5}
\definecolor{deepred}{rgb}{0.6,0,0}
\definecolor{deepgreen}{rgb}{0,0.5,0}
%
%% Code printing
\lstset{
  basicstyle={\ttfamily},
  identifierstyle={\small},
  %commentstyle={\small\itshape\color{deepgreen}},
  commentstyle={\small\color{deepgreen}},
  keywordstyle={\small\bfseries\color{deepblue}},
  ndkeywordstyle={\small},
  stringstyle={\small\ttfamily\color{deepred}},
  tabsize=4,
  frame={tb},
  breaklines=true,
  columns=[l]{fullflexible},
  numbers=left,
  xrightmargin=0zw,
  xleftmargin=3zw,
  numberstyle={\scriptsize},
  stepnumber=1,
  numbersep=1zw,
  lineskip=-0.5ex
}
%
%% Color Theme
%%   cf) Frankfurt,AnnArbor, Antibes, Berlin,
%%       Berkeley, Bergen, Boadilla, boxes, Copenhagen
\usetheme{AnnArbor}
%
%% Label number of figures, tables or codes
\setbeamertemplate{caption}[numbered]
%% Remove icons of lower right.
\setbeamertemplate{navigation symbols}{}
%
% 箇条書きを段階的にみせたいとき
%\beamerdefaultoverlayspecification{<+->}
% 隠してるアイテムを半透明で表示
%\setbeamercovered{transparent}
%
%% itemize, enumerate, description Style
%\setbeamercolor{itemize item}{fg=red}
%\setbeamercolor{itemize subitem}{fg=red}
\setbeamertemplate{itemize items}[circle]
\setbeamercolor{enumerate items}{fg=deepblue}
\setbeamertemplate{enumerate items}{\insertenumlabel} %<---- appearence of number in second level
\setbeamertemplate{enumerate subitem}{\insertenumlabel.\insertsubenumlabel} %<---- appearence of number in second level
%\setbeamercolor{description item}{fg=deepgreen}
%\setbeamercolor{description items}{fg=deepgreen}
\setbeamercolor{description item}{fg=deepblue}
\setbeamercolor{description items}{fg=deepblue}
%
%% 目次
\setbeamertemplate{section in toc}
  {\leavevmode\leftskip=2ex\inserttocsectionnumber. \inserttocsection\par}
\setbeamertemplate{subsection in toc}
  {\leavevmode\leftskip=2em$\bullet$\hskip1em\inserttocsubsection\par}
%
%% Block
\setbeamercolor{block title}{fg=black, bg=blue!40!white}
\setbeamercolor{block body}{fg=black, bg=blue!5!white}
\setbeamercolor{block title alerted}{fg=black, bg=red!40!white}
\setbeamercolor{block body  alerted}{fg=black, bg=red!5!white}
\setbeamercolor{block title example}{fg=black, bg=green!40!white}
\setbeamercolor{block body  example}{fg=black, bg=green!5!white}
\setbeamerfont{block title}{size=\large}
\setbeamerfont{block body}{size=\normalsize}
\setbeamerfont{block title alerted}{size=\large}
\setbeamerfont{block body  alerted}{size=\normalsize}
\setbeamerfont{block title example}{size=\large}
\setbeamerfont{block body  example}{size=\normalsize}
%
%% Use divergence
\newcommand{\divergence}{\mathrm{div}\,}
%% Use gradient
\newcommand{\grad}{\mathrm{grad}\,}
%% Use rotation
\newcommand{\rot}{\mathrm{rot}\,}
%
%% Use date for Japanese
\newcommand{\todayJP}{\number\year 年\number\month 月\number\day 日}
%
% The caption label for it.
\renewcommand{\figurename}{図}
\renewcommand{\tablename}{表}
\renewcommand{\lstlistingname}{コード}
%
% Japanese font changes Gothic Font.
\renewcommand{\kanjifamilydefault}{\gtdefault}
%
%% Remove Header
%\setbeamertemplate{headline}{}
%
